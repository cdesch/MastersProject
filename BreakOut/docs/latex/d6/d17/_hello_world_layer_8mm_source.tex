\hypertarget{_hello_world_layer_8mm}{\section{Hello\-World\-Layer.\-mm}
\label{d6/d17/_hello_world_layer_8mm}\index{Break\-Out/\-Hello\-World\-Layer.\-mm@{Break\-Out/\-Hello\-World\-Layer.\-mm}}
}

\begin{DoxyCode}
00001 \textcolor{comment}{//}
00002 \textcolor{comment}{//  HelloWorldLayer.mm}
00003 \textcolor{comment}{//  LeapPuzz}
00004 \textcolor{comment}{//}
00005 \textcolor{comment}{//  Created by cj on 2/3/13.}
00006 \textcolor{comment}{//  Copyright \_\_MyCompanyName\_\_ 2013. All rights reserved.}
00007 \textcolor{comment}{//}
00008 
00009 \textcolor{comment}{// Import the interfaces}
00010 \textcolor{preprocessor}{#import "HelloWorldLayer.h"}
00011 \textcolor{preprocessor}{#import "PhysicsSprite.h"}
00012 \textcolor{comment}{//Pixel to metres ratio. Box2D uses metres as the unit for measurement.}
00013 \textcolor{comment}{//This ratio defines how many pixels correspond to 1 Box2D "metre"}
00014 \textcolor{comment}{//Box2D is optimized for objects of 1x1 metre therefore it makes sense}
00015 \textcolor{comment}{//to define the ratio so that your most common object type is 1x1 metre.}
00016 \textcolor{preprocessor}{#define PTM\_RATIO 32}
00017 \textcolor{preprocessor}{}
00018 \textcolor{keyword}{enum} \{
00019     kTagParentNode = 1,
00020 \};
00021 
00022 
00023 
00024 \textcolor{preprocessor}{#pragma mark - HelloWorldLayer}
00025 \textcolor{preprocessor}{}
\hypertarget{_hello_world_layer_8mm_source_l00026}{}\hyperlink{category_hello_world_layer_07_08}{00026} \textcolor{keyword}{@interface }\hyperlink{interface_hello_world_layer}{HelloWorldLayer}()
00027 -(void) initPhysics;
00028 -(void) addNewSpriteAtPosition:(CGPoint)p;
00029 -(void) createResetButton;
00030 \textcolor{keyword}{@end}
00031 
00032 \textcolor{keyword}{@implementation }\hyperlink{interface_hello_world_layer}{HelloWorldLayer}
00033 
00034 -(id) init
00035 \{
00036     \textcolor{keywordflow}{if}( (\textcolor{keyword}{self}=[super init])) \{
00037         
00038         \textcolor{comment}{// enable events}
00039         
00040 \textcolor{preprocessor}{#ifdef \_\_IPHONE\_OS\_VERSION\_MAX\_ALLOWED}
00041 \textcolor{preprocessor}{}        \textcolor{keyword}{self}.isTouchEnabled = YES;
00042         \textcolor{keyword}{self}.isAccelerometerEnabled = YES;
00043 \textcolor{preprocessor}{#elif defined(\_\_MAC\_OS\_X\_VERSION\_MAX\_ALLOWED)}
00044 \textcolor{preprocessor}{}        \textcolor{keyword}{self}.isMouseEnabled = YES;
00045 \textcolor{preprocessor}{#endif}
00046 \textcolor{preprocessor}{}        CGSize s = [CCDirector sharedDirector].winSize;
00047         
00048         \textcolor{comment}{// init physics}
00049         [\textcolor{keyword}{self} initPhysics];
00050         
00051         \textcolor{comment}{// create reset button}
00052         [\textcolor{keyword}{self} createResetButton];
00053         
00054         \textcolor{comment}{//Set up sprite}
00055         
00056 \textcolor{preprocessor}{#if 1}
00057 \textcolor{preprocessor}{}        \textcolor{comment}{// Use batch node. Faster}
00058         CCSpriteBatchNode *parent = [CCSpriteBatchNode batchNodeWithFile:@"blocks.png" capacity:100];
00059         spriteTexture\_ = [parent texture];
00060 \textcolor{preprocessor}{#else}
00061 \textcolor{preprocessor}{}        \textcolor{comment}{// doesn't use batch node. Slower}
00062         spriteTexture\_ = [[CCTextureCache sharedTextureCache] addImage:@"blocks.png"];
00063         CCNode *parent = [CCNode node];
00064 \textcolor{preprocessor}{#endif}
00065 \textcolor{preprocessor}{}        [\textcolor{keyword}{self} addChild:parent z:0 tag:kTagParentNode];
00066         
00067         
00068         [\textcolor{keyword}{self} addNewSpriteAtPosition:ccp(s.width/2, s.height/2)];
00069         
00070         CCLabelTTF *label = [CCLabelTTF labelWithString:@"LeapPuzz" fontName:@"Marker Felt" fontSize:32];
00071         [\textcolor{keyword}{self} addChild:label z:0];
00072         [label setColor:ccc3(0,0,255)];
00073         label.position = ccp( s.width/2, s.height-50);
00074         
00075         [\textcolor{keyword}{self} scheduleUpdate];
00076         
00077         trackableList = [[NSMutableDictionary alloc] init];
00078         
00079         [\textcolor{keyword}{self} run];
00080         
00081 
00082     \}
00083     \textcolor{keywordflow}{return} \textcolor{keyword}{self};
00084 \}
00085 
00086 - (void)run
00087 \{
00088     controller = [[LeapController alloc] init];
00089     [controller addDelegate:self];
00090     NSLog(\textcolor{stringliteral}{@"running"});
00091 \}
00092 
00093 \textcolor{preprocessor}{#pragma mark - SampleDelegate Callbacks}
00094 \textcolor{preprocessor}{}
00095 - (void)onInit:(LeapController *)aController
00096 \{
00097     NSLog(\textcolor{stringliteral}{@"Initialized"});
00098 \}
00099 
00100 - (void)onConnect:(LeapController *)aController
00101 \{
00102     NSLog(\textcolor{stringliteral}{@"Connected"});
00103 \}
00104 
00105 - (void)onDisconnect:(LeapController *)aController
00106 \{
00107     NSLog(\textcolor{stringliteral}{@"Disconnected"});
00108 \}
00109 
00110 - (void)onExit:(LeapController *)aController
00111 \{
00112     NSLog(\textcolor{stringliteral}{@"Exited"});
00113 \}
00114 
00115 - (void)onFrame:(LeapController *)aController
00116 \{
00117     \textcolor{comment}{// Get the most recent frame and report some basic information}
00118     LeapFrame *frame = [aController frame:0];
00119     \textcolor{comment}{/*}
00120 \textcolor{comment}{    NSLog(@"Frame id: %lld, timestamp: %lld, hands: %ld, fingers: %ld, tools: %ld",}
00121 \textcolor{comment}{          [frame id], [frame timestamp], [[frame hands] count],}
00122 \textcolor{comment}{          [[frame fingers] count], [[frame tools] count]);}
00123 \textcolor{comment}{    }
00124 \textcolor{comment}{     */}
00125     \textcolor{keywordflow}{if} ([[frame hands] count] != 0) \{
00126         \textcolor{comment}{// Get the first hand}
00127         LeapHand *hand = [[frame hands] objectAtIndex:0];
00128 
00129         
00130         \textcolor{comment}{// Check if the hand has any fingers}
00131         NSArray *fingers = [hand fingers];
00132         
00133         \textcolor{keywordflow}{if} ([fingers count] != 0) \{
00134             
00135             \textcolor{comment}{// Calculate the hand's average finger tip position}
00136             LeapVector *avgPos = [[LeapVector alloc] init];
00137             \textcolor{keywordflow}{for} (\textcolor{keywordtype}{int} i = 0; i < [fingers count]; i++) \{
00138                 LeapFinger *finger = [fingers objectAtIndex:i];
00139                 avgPos = [avgPos plus:[finger tipPosition]];
00140                 
00141                 
00142                 NSString* fingerID = [NSString stringWithFormat:@"%d", finger.id];
00143                 
00144                 \textcolor{comment}{//Check if the Finger ID exists in the list already}
00145                 \textcolor{keywordflow}{if} ([trackableList objectForKey:fingerID]) \{
00146                     
00147                     \textcolor{comment}{//If it does exist update the position on the screen}
00148                     RedDot* sprite = [trackableList objectForKey:fingerID];
00149                     sprite.position = [\textcolor{keyword}{self} covertLeapCoordinates:CGPointMake(finger.tipPosition.x, finger.
      tipPosition.y)];
00150                     sprite.updated = TRUE;
00151                     
00152                     
00153                 \}\textcolor{keywordflow}{else}\{
00154                     
00155                     NSLog(\textcolor{stringliteral}{@"x %0.0f y %0.0f z %0.0f"}, finger.tipPosition.x, finger.tipPosition.y, finger.
      tipPosition.z);
00156                    \textcolor{comment}{// CGPoint point = [[CCDirector sharedDirector] convertEventToGL:event];}
00157                     \textcolor{comment}{//CGPoint mouseLocation = [self convertToNodeSpace:point];}
00158                 
00159                     \textcolor{comment}{//Add it to the dictionary}
00160                     RedDot* redDot = [\textcolor{keyword}{self} addRedDot:CGPointMake(finger.tipPosition.x, finger.tipPosition.y
      ) finger:fingerID];
00161                     [trackableList setObject:redDot forKey:fingerID];
00162                 \}
00163             \}
00164             
00165             avgPos = [avgPos divide:[fingers count]];
00166             
00167             \textcolor{comment}{//NSLog(@"Hand has %ld fingers, average finger tip position %@", [fingers count], avgPos);}
00168             \textcolor{keywordflow}{for} (LeapFinger* finger in fingers)\{
00169                 
00170                 \textcolor{comment}{//NSLog(@"Finger ID %d %ld", finger.id, (unsigned long)[finger hash]);}
00171             \}
00172             
00173         \}
00174         
00175         \textcolor{comment}{//}
00176         [\textcolor{keyword}{self} checkFingerExists];
00177         
00178         \textcolor{comment}{// Get the hand's sphere radius and palm position}
00179         \textcolor{comment}{/*}
00180 \textcolor{comment}{        NSLog(@"Hand sphere radius: %f mm, palm position: %@",}
00181 \textcolor{comment}{              [hand sphereRadius], [hand palmPosition]);}
00182 \textcolor{comment}{        */}
00183         \textcolor{comment}{// Get the hand's normal vector and direction}
00184         \textcolor{keyword}{const} LeapVector *normal = [hand palmNormal];
00185         \textcolor{keyword}{const} LeapVector *direction = [hand direction];
00186         
00187         \textcolor{comment}{/*}
00188 \textcolor{comment}{        // Calculate the hand's pitch, roll, and yaw angles}
00189 \textcolor{comment}{        NSLog(@"Hand pitch: %f degrees, roll: %f degrees, yaw: %f degrees\(\backslash\)n",}
00190 \textcolor{comment}{              [direction pitch] * LEAP\_RAD\_TO\_DEG,}
00191 \textcolor{comment}{              [normal roll] * LEAP\_RAD\_TO\_DEG,}
00192 \textcolor{comment}{              [direction yaw] * LEAP\_RAD\_TO\_DEG);}
00193 \textcolor{comment}{         */}
00194     \}
00195 \}
00196 
00197 
00198 - (void)moveRedDot\{
00199     
00200 
00201 \}
00202 
00203 \textcolor{comment}{//Cycle through all the trackable dots and check if the fingers still exist.}
00204 \textcolor{comment}{//If they don't, delete them.}
00205 - (void)checkFingerExists\{
00206     
00207     \textcolor{keywordflow}{for} (\textcolor{keywordtype}{id} key in [trackableList allKeys]) \{
00208         RedDot* sprite = [trackableList objectForKey:key];
00209         \textcolor{keywordflow}{if} (sprite.updated) \{
00210             sprite.updated = FALSE;
00211             \textcolor{keywordflow}{return};
00212         \}\textcolor{keywordflow}{else}\{
00213             CCNode *parent = [\textcolor{keyword}{self} getChildByTag:kTagParentNode];
00214             [trackableList removeObjectForKey:key];
00215             [parent removeChild:sprite cleanup:YES];
00216             
00217         \}
00218     \}
00219 \}
00220 
00221 
00222 \textcolor{preprocessor}{#pragma mark -}
00223 \textcolor{preprocessor}{}
00224 -(void) createResetButton
00225 \{
00226     CCMenuItemLabel *reset = [CCMenuItemFont itemWithString:@"Reset" block:^(id sender)\{
00227         CCScene *s = [CCScene node];
00228         id child = [\hyperlink{interface_hello_world_layer}{HelloWorldLayer} node];
00229         [s addChild:child];
00230         [[CCDirector sharedDirector] replaceScene: s];
00231     \}];
00232     
00233     CCMenu *menu = [CCMenu menuWithItems:reset, nil];
00234     
00235     CGSize s = [[CCDirector sharedDirector] winSize];
00236     
00237     menu.position = ccp(s.width/2, 30);
00238     [\textcolor{keyword}{self} addChild: menu z:-1]; 
00239     
00240 \}
00241 
00242 -(void) initPhysics
00243 \{
00244     
00245     CGSize s = [[CCDirector sharedDirector] winSize];
00246     
00247     \textcolor{comment}{//Gravity}
00248     b2Vec2 gravity;
00249     gravity.Set(0.0f, 0.0f);
00250     world = \textcolor{keyword}{new} b2World(gravity);
00251     
00252     
00253     \textcolor{comment}{// Do we want to let bodies sleep?}
00254     world->SetAllowSleeping(\textcolor{keyword}{true});
00255     
00256     world->SetContinuousPhysics(\textcolor{keyword}{true});
00257     
00258     m\_debugDraw = \textcolor{keyword}{new} \hyperlink{class_g_l_e_s_debug_draw}{GLESDebugDraw}( PTM\_RATIO );
00259     world->SetDebugDraw(m\_debugDraw);
00260     
00261     \_world = world;
00262     
00263     uint32 flags = 0;
00264     flags += b2Draw::e\_shapeBit;
00265     \textcolor{comment}{//      flags += b2Draw::e\_jointBit;}
00266     \textcolor{comment}{//      flags += b2Draw::e\_aabbBit;}
00267     \textcolor{comment}{//      flags += b2Draw::e\_pairBit;}
00268     \textcolor{comment}{//      flags += b2Draw::e\_centerOfMassBit;}
00269     m\_debugDraw->SetFlags(flags);       
00270     
00271     
00272     \textcolor{comment}{// Define the ground body.}
00273     b2BodyDef groundBodyDef;
00274     groundBodyDef.position.Set(0, 0); \textcolor{comment}{// bottom-left corner}
00275     
00276     \textcolor{comment}{// Call the body factory which allocates memory for the ground body}
00277     \textcolor{comment}{// from a pool and creates the ground box shape (also from a pool).}
00278     \textcolor{comment}{// The body is also added to the world.}
00279     b2Body* groundBody = world->CreateBody(&groundBodyDef);
00280     
00281     \textcolor{comment}{// Define the ground box shape.}
00282     b2EdgeShape groundBox;      
00283     
00284     \textcolor{comment}{// bottom}
00285     
00286     groundBox.Set(b2Vec2(0,0), b2Vec2(s.width/PTM\_RATIO,0));
00287     groundBody->CreateFixture(&groundBox,0);
00288     
00289     \textcolor{comment}{// top}
00290     groundBox.Set(b2Vec2(0,s.height/PTM\_RATIO), b2Vec2(s.width/PTM\_RATIO,s.height/PTM\_RATIO));
00291     groundBody->CreateFixture(&groundBox,0);
00292     
00293     \textcolor{comment}{// left}
00294     groundBox.Set(b2Vec2(0,s.height/PTM\_RATIO), b2Vec2(0,0));
00295     groundBody->CreateFixture(&groundBox,0);
00296     
00297     \textcolor{comment}{// right}
00298     groundBox.Set(b2Vec2(s.width/PTM\_RATIO,s.height/PTM\_RATIO), b2Vec2(s.width/PTM\_RATIO,0));
00299     groundBody->CreateFixture(&groundBox,0);
00300     
00301     \_groundBody = groundBody;
00302 \}
00303 
00304 -(void) draw
00305 \{
00306     \textcolor{comment}{//}
00307     \textcolor{comment}{// IMPORTANT:}
00308     \textcolor{comment}{// This is only for debug purposes}
00309     \textcolor{comment}{// It is recommend to disable it}
00310     \textcolor{comment}{//}
00311     [\textcolor{keyword}{super} draw];
00312     
00313     ccGLEnableVertexAttribs( kCCVertexAttribFlag\_Position );
00314     
00315     kmGLPushMatrix();
00316     
00317     world->DrawDebugData(); 
00318     
00319     kmGLPopMatrix();
00320 \}
00321 
00322 - (RedDot*)addRedDot:(CGPoint)p finger:(NSString*)fingerID\{
00323     CCNode *parent = [\textcolor{keyword}{self} getChildByTag:kTagParentNode];
00324     \textcolor{keywordtype}{int} idx = (CCRANDOM\_0\_1() > .5 ? 0:1);
00325     \textcolor{keywordtype}{int} idy = (CCRANDOM\_0\_1() > .5 ? 0:1);
00326     
00327     \textcolor{comment}{//RedDot *sprite = [RedDot spriteWithFile:@"redcrosshair.png"];}
00328     RedDot *sprite = [RedDot spriteWithTexture:spriteTexture\_ rect:CGRectMake(32 * idx,32 * idy,32,32)];        
00329     [parent addChild:sprite];
00330     sprite.updated = TRUE;
00331     sprite.fingerID = fingerID;
00332     sprite.position = ccp( p.x, p.y);
00333     
00334     \textcolor{keywordflow}{return} sprite;
00335 \}
00336 
00337 - (CGPoint)covertLeapCoordinates:(CGPoint)p\{
00338 
00339     CGSize s = [[CCDirector sharedDirector] winSize];
00340     \textcolor{keywordtype}{float} screenCenter = 0.0f;
00341     \textcolor{keywordtype}{float} xScale = 1.75f;
00342     \textcolor{keywordtype}{float} yScale = 1.25f;
00343     \textcolor{keywordflow}{return} CGPointMake((s.width/2)+ (( p.x - screenCenter) * xScale), p.y * yScale);
00344 \}
00345 
00346 -(void) addNewSpriteAtPosition:(CGPoint)p
00347 \{
00348     CCLOG(\textcolor{stringliteral}{@"Add sprite %0.2f x %02.f"},p.x,p.y);
00349     CCNode *parent = [\textcolor{keyword}{self} getChildByTag:kTagParentNode];
00350     
00351     \textcolor{comment}{//We have a 64x64 sprite sheet with 4 different 32x32 images.  The following code is}
00352     \textcolor{comment}{//just randomly picking one of the images}
00353     \textcolor{keywordtype}{int} idx = (CCRANDOM\_0\_1() > .5 ? 0:1);
00354     \textcolor{keywordtype}{int} idy = (CCRANDOM\_0\_1() > .5 ? 0:1);
00355     \hyperlink{interface_physics_sprite}{PhysicsSprite} *sprite = [\hyperlink{interface_physics_sprite}{PhysicsSprite} spriteWithTexture:spriteTexture\_ rect:
      CGRectMake(32 * idx,32 * idy,32,32)];                        
00356     [parent addChild:sprite];
00357     sprite.position = [\textcolor{keyword}{self} covertLeapCoordinates:p];
00358     \textcolor{comment}{//sprite.position = ccp( p.x, p.y);}
00359     
00360     \textcolor{comment}{// Define the dynamic body.}
00361     \textcolor{comment}{//Set up a 1m squared box in the physics world}
00362     b2BodyDef bodyDef;
00363     bodyDef.type = b2\_dynamicBody;
00364     bodyDef.position.Set(p.x/PTM\_RATIO, p.y/PTM\_RATIO);
00365 
00366     \textcolor{comment}{//bodyDef.userData  = (void *) CFBridgingRetain(sprite);}
00367     bodyDef.userData  = (\_\_bridge \textcolor{keywordtype}{void} *)sprite;
00368     b2Body *body = world->CreateBody(&bodyDef);
00369     
00370     \textcolor{comment}{// Define another box shape for our dynamic body.}
00371     b2PolygonShape dynamicBox;
00372     dynamicBox.SetAsBox(.5f, .5f);\textcolor{comment}{//These are mid points for our 1m box}
00373     
00374     \textcolor{comment}{// Define the dynamic body fixture.}
00375     b2FixtureDef fixtureDef;
00376     fixtureDef.shape = &dynamicBox; 
00377     fixtureDef.density = 1.0f;
00378     fixtureDef.friction = 0.3f;
00379     body->CreateFixture(&fixtureDef);
00380     
00381     [sprite setPhysicsBody:body];
00382 \}
00383 
00384 -(void) addPieceAtPosition:(CGPoint)p
00385 \{
00386     CCLOG(\textcolor{stringliteral}{@"Add sprite %0.2f x %02.f"},p.x,p.y);
00387     CCNode *parent = [\textcolor{keyword}{self} getChildByTag:kTagParentNode];
00388     
00389     \textcolor{comment}{//We have a 64x64 sprite sheet with 4 different 32x32 images.  The following code is}
00390     \textcolor{comment}{//just randomly picking one of the images}
00391     \textcolor{keywordtype}{int} idx = (CCRANDOM\_0\_1() > .5 ? 0:1);
00392     \textcolor{keywordtype}{int} idy = (CCRANDOM\_0\_1() > .5 ? 0:1);
00393     \hyperlink{interface_physics_sprite}{PhysicsSprite} *sprite = [\hyperlink{interface_physics_sprite}{PhysicsSprite} spriteWithTexture:spriteTexture\_ rect:
      CGRectMake(32 * idx,32 * idy,32,32)];
00394     [parent addChild:sprite];
00395     
00396     sprite.position = ccp( p.x, p.y);
00397     
00398     \textcolor{comment}{// Define the dynamic body.}
00399     \textcolor{comment}{//Set up a 1m squared box in the physics world}
00400     b2BodyDef bodyDef;
00401     bodyDef.type = b2\_dynamicBody;
00402     bodyDef.position.Set(p.x/PTM\_RATIO, p.y/PTM\_RATIO);
00403     b2Body *body = world->CreateBody(&bodyDef);
00404     
00405     \textcolor{comment}{// Define another box shape for our dynamic body.}
00406     b2PolygonShape dynamicBox;
00407     dynamicBox.SetAsBox(.5f, .5f);\textcolor{comment}{//These are mid points for our 1m box}
00408     
00409     \textcolor{comment}{// Define the dynamic body fixture.}
00410     b2FixtureDef fixtureDef;
00411     fixtureDef.shape = &dynamicBox;
00412     fixtureDef.density = 1.0f;
00413     fixtureDef.friction = 0.3f;
00414     body->CreateFixture(&fixtureDef);
00415     
00416     [sprite setPhysicsBody:body];
00417 \}
00418 
00419 -(void) update: (ccTime) dt
00420 \{
00421     \textcolor{comment}{//It is recommended that a fixed time step is used with Box2D for stability}
00422     \textcolor{comment}{//of the simulation, however, we are using a variable time step here.}
00423     \textcolor{comment}{//You need to make an informed choice, the following URL is useful}
00424     \textcolor{comment}{//http://gafferongames.com/game-physics/fix-your-timestep/}
00425     
00426     int32 velocityIterations = 8;
00427     int32 positionIterations = 1;
00428     
00429     \textcolor{comment}{// Instruct the world to perform a single step of simulation. It is}
00430     \textcolor{comment}{// generally best to keep the time step and iterations fixed.}
00431     world->Step(dt, velocityIterations, positionIterations);    
00432 \}
00433 
00434 \textcolor{preprocessor}{#ifdef \_\_IPHONE\_OS\_VERSION\_MAX\_ALLOWED}
00435 \textcolor{preprocessor}{}
00436 - (void)ccTouchesEnded:(NSSet *)touches withEvent:(UIEvent *)event
00437 \{
00438     \textcolor{comment}{//Add a new body/atlas sprite at the touched location}
00439     \textcolor{keywordflow}{for}( UITouch *touch in touches ) \{
00440         CGPoint location = [touch locationInView: [touch view]];
00441         
00442         location = [[CCDirector sharedDirector] convertToGL: location];
00443         
00444         [\textcolor{keyword}{self} addNewSpriteAtPosition: location];
00445         
00446         
00447     \}
00448 \}
00449 
00450 \textcolor{preprocessor}{#elif defined(\_\_MAC\_OS\_X\_VERSION\_MAX\_ALLOWED)}
00451 \textcolor{preprocessor}{}\textcolor{comment}{/*}
00452 \textcolor{comment}{- (BOOL)ccTouchBegan:(UITouch *)touch withEvent:(UIEvent *)event \{}
00453 \textcolor{comment}{    CGPoint touchLocation = [self convertTouchToNodeSpace:touch];}
00454 \textcolor{comment}{    [self selectSpriteForTouch:touchLocation];}
00455 \textcolor{comment}{    return TRUE;}
00456 \textcolor{comment}{\}}
00457 \textcolor{comment}{ */}
00458 
00459 
00460 
00461 
00462 \textcolor{preprocessor}{#pragma mark - Touch Handling}
00463 \textcolor{preprocessor}{}
00464 - (BOOL) ccMouseDown:(NSEvent *)event\{
00465     
00466     \textcolor{keywordflow}{if} (\_mouseJoint != NULL) \textcolor{keywordflow}{return} NO;
00467     
00468 
00469     CGPoint point = [[CCDirector sharedDirector] convertEventToGL:event];
00470     CGPoint mouseLocation = [\textcolor{keyword}{self} convertToNodeSpace:point];
00471     CGPoint translation = (mouseLocation);
00472     CGPoint location = translation;
00473     \textcolor{comment}{//location = [[CCDirector sharedDirector] convertToGL:location];}
00474     b2Vec2 locationWorld = b2Vec2(location.x/PTM\_RATIO, location.y/PTM\_RATIO);
00475     
00476     
00477     \textcolor{comment}{// Loop through all of the Box2D bodies in our Box2D world..}
00478     \textcolor{keywordflow}{for}(b2Body *b = \_world->GetBodyList(); b; b=b->GetNext()) \{
00479         
00480         
00481         \textcolor{comment}{// See if there's any user data attached to the Box2D body}
00482         \textcolor{comment}{// There should be, since we set it in addBoxBodyForSprite}
00483         
00484         \textcolor{keywordflow}{if} (b->GetUserData() != NULL) \{
00485             \textcolor{comment}{// We know that the user data is a sprite since we set}
00486             \textcolor{comment}{// it that way, so cast it...}
00487             
00488             \textcolor{comment}{//PhysicsSprite *sprite = (PhysicsSprite *)CFBridgingRelease(b->GetUserData());}
00489             
00490             
00491             \textcolor{keywordflow}{for}(b2Fixture *fixture = b->GetFixtureList(); fixture; fixture=fixture->GetNext()) \{
00492                 
00493                 \textcolor{keywordflow}{if}(fixture->TestPoint(locationWorld))\{
00494                     \textcolor{comment}{//NSLog(@"Touched itemType %d", sprite.itemType);}
00495                     b2MouseJointDef md;
00496                     md.bodyA = \_groundBody;
00497                     md.bodyB = b;
00498                     md.target = locationWorld;
00499                     md.collideConnected = \textcolor{keyword}{true};
00500                     md.maxForce = 1000.0f * b->GetMass();
00501                     
00502                     \_mouseJoint = (b2MouseJoint *)\_world->CreateJoint(&md);
00503                     b->SetAwake(\textcolor{keyword}{true});
00504                 \}\textcolor{keywordflow}{else}\{
00505                     \textcolor{comment}{//NSLog(@"NOT TOUCHED");}
00506                 \}
00507             \}
00508         \}
00509     \}
00510     \textcolor{keywordflow}{return} YES;
00511 \}
00512 
00513 - (BOOL)ccMouseDragged:(NSEvent *)event \{
00514     
00515     \textcolor{keywordflow}{if} (\_mouseJoint == NULL) \textcolor{keywordflow}{return} NO;
00516     
00517 
00518     CGPoint point = [[CCDirector sharedDirector] convertEventToGL:event];
00519     CGPoint mouseLocation = [\textcolor{keyword}{self} convertToNodeSpace:point];
00520     CGPoint translation = (mouseLocation);
00521     CGPoint location = translation;
00522     \textcolor{comment}{//location = [[CCDirector sharedDirector] convertToGL:location];}
00523     b2Vec2 locationWorld = b2Vec2(location.x/PTM\_RATIO, location.y/PTM\_RATIO);
00524     
00525     \_mouseJoint->SetTarget(locationWorld);
00526     
00527     \textcolor{keywordflow}{return} YES;
00528     
00529 \}
00530 
00531 - (BOOL)ccMouseUp:(NSEvent *)event\{
00532     \textcolor{keywordflow}{if} (\_mouseJoint) \{
00533         \_world->DestroyJoint(\_mouseJoint);
00534         \_mouseJoint = NULL;
00535         
00536         \textcolor{comment}{//Check for any dangling mouse joints}
00537         \textcolor{keywordflow}{if}(\_world->GetJointCount() > 0)\{
00538             \textcolor{comment}{//NSLog(@"Found %d Extra Joints", \_world->GetJointCount() );}
00539             \textcolor{keywordflow}{for}(b2Joint *b = \_world->GetJointList(); b; b=b->GetNext()) \{
00540                 \textcolor{comment}{//NSLog(@"Destproying the Dangling Joint");}
00541                 \textcolor{comment}{//Should check type first}
00542                 \textcolor{keywordflow}{if}(b)\{
00543                     \_world->DestroyJoint(b);
00544                     b = NULL;
00545                     \textcolor{keywordflow}{return} YES;
00546                 \}
00547             \}
00548         \}
00549     \}\textcolor{keywordflow}{else}\{
00550         
00551         
00552         CGPoint point = [[CCDirector sharedDirector] convertEventToGL:event];
00553         CGPoint mouseLocation = [\textcolor{keyword}{self} convertToNodeSpace:point];
00554         CGPoint translation = (mouseLocation);
00555         CGPoint location = translation;
00556     
00557         
00558         [\textcolor{keyword}{self} addNewSpriteAtPosition: location];
00559         
00560     \}
00561     \textcolor{keywordflow}{return} YES;
00562 \}
00563 
00564 \textcolor{preprocessor}{#endif}
00565 \textcolor{preprocessor}{}
00566 \textcolor{keyword}{@end}
\end{DoxyCode}
