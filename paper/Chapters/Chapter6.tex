% Chapter 1

\chapter{Not named Chapter} % Main chapter title

\label{Chapter5} % For referencing the chapter elsewhere, use \ref{Chapter1} 

\lhead{Chapter 5. \emph{Not named chapter}} % This is for the header on each page - perhaps a shortened title

%----------------------------------------------------------------------------------------

\subsection{Observations}
The children would first sketch an outline of their drawing and then attempt to color in their outlines. This proved difficult in the case of the LeapMotion. Drawing the lines proved fairly easy for the children while shading in sections appeared more difficult. I found the opposite to be true of my own experience as the lines were harder to draw than shading in the areas.

The LeapMotion is so sensitive that it often has to be averaged out to reduce the noise in which is received from raw data. 

\section{Similar Applications}

Comparison to industry competitors Corel's Painter Freestyle which will have many of the same features. In terms of interface design they chose a similar layout of control mechanisms. We haven't be able to to see this all quiet yet since it has not been released to perform a full comparison although from the initial details given on their website it seems that their application could be similar to ours in many respects. \cite{corel}

\section{Google Earth}
Examples of dedication are shown in some example applications where the LeapMotion has a specific purpose. Google Earth uses it for navigation only allowing the user to pan, rotate and elevated the camera in relation to the earth. Interpreting the hand motions as the path of airplane as the control mechanism with yaw, pitch, roll, bank and elevation. 

\section{Summary}
With work extending into different areas of how the leapmotion could be used to control
<FIX>
\section{Future Research}
Some Future Research go here <FIX>

\section{Acknowledgments}
We would like to thank Montclair State University and our child design partners forming KidsTeam.

\section{Afterward}
Collaborative design with only one LeapMotion proved to be difficult as only one child could use the LeapMotion at a time. This was disappointing in some respects. <Fix>


