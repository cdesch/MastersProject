% Chapter 1

\chapter{Summary} % Main chapter title

\label{Chapter5} % For referencing the chapter elsewhere, use \ref{Chapter1} 

\lhead{Chapter 5. \emph{Summary}} % This is for the header on each page - perhaps a shortened title

%----------------------------------------------------------------------------------------

\section{General Observations}
The children would first sketch an outline of their drawing and then attempt to color in their outlines. This proved difficult in the case of the LeapMotion. Drawing the lines proved fairly easy for the children while shading in sections appeared more difficult. I found the opposite to be true of my own experience as the lines were harder to draw than shading in the areas.

\section{Similar Applications}

Comparison to industry competitors Corel's Painter Freestyle which will have many of the same features. In terms of interface design they chose a similar layout of control mechanisms. We haven't be able to to see this all quiet yet since it has not been released to perform a full comparison although from the initial details given on their website seems to lead that their application could be similar to ours in many respects. \cite{corel}

\subsection{Google Earth}
Examples of dedication are shown in some example applications where the LeapMotion has a specific purpose. Google Earth uses it for navigation only allowing the user to pan, rotate and elevated the camera in relation to the earth. Interpreting the hand motions as the path of airplane as the control mechanism with yaw, pitch, roll, bank and elevation. 

The way of controlling the Google Earth is similar to the idea of controlling the Templerun game brainstormed in from Session 1~\ref{session1}. This connection was not apparent when first considering the idea for controlling the game. 

\section{Conclusions}
The LeapMotion is a great device for capturing the motions and positions of the hand in real time but is tough to consider as replacement for the keyboard and mouse. It is foreseeable that the main application for the LeapMotion will not be as a general use device but for specific applications which enable expressive movement. Games, drawing and music applications are good examples of where the LeapMotion can focus on specialized actions. Integration into general purpose applications will be difficult due to existing designs and interface paradigms. 

\section{Licenses}


%Integration with other devices such as mounting on a keyboard or embedding into a laptop or monitor will 


%%In the initial phases terminology cocolily synomous with the mouse was avoided. 


